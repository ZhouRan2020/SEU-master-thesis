\chapter{绪论}
\label{chp:installation}
本章介绍低轨卫星通信系统资源分配方法的研究背景及国内外研究现状。\par
\nomenclature{PDF}{Portable Document Format}
\section{研究背景}
\label{sec:tex_environment}
1945年10月,英国军官阿瑟克拉克发表了《地球外的中继站》一文,首次提出了利用
太空中三颗对地静止轨道卫星,实现覆盖全球的无线通信的设想。卫星通信是地球站之
间利用空间站转发或反射的无线电通信\cite{application2015min}。得益于覆盖范
围广、通信距离远、通信容量大、通信线路稳定、支持多址联接、建设周期短、组网灵
活等优势,卫星通信自提出后的七十余年以来,发展出了卫星固定业务、卫星移动业务
、卫星广播业务、卫星移动广播业务、卫星跟踪和数据中继业务五大通信业务,展现出
蓬勃的活力。\par

通信和宇航技术的发展,推动了卫星通信技术的进步。现代卫星技术已经可以发送重达
数吨的通信卫星;另一方面,智能手持终端已经可以在偏远地区直接接收卫星信号。因
此,静止轨道卫星通信系统呈现出大卫星小终端的发展趋势。有别于卫星通信发展早期,
通信业务和通信网络已不再分类明确。卫星固定、卫星移动、卫星广播三种业务相互融
合;地面电信网、计算机网和有线电视网三种网络相互融合已经是不可逆的趋势。相较
于追求简单和可靠的静止轨道卫星,低轨道卫星系统路径损耗小、时延低,可以实现多
重覆盖、高抗毁性和星上路由,能够满足日益增长、不断变化的通信需求,是学术和工
程上的重点研究对象。\par%轨道

尽管通信卫星的性能已有显著提升,但是地面站的通信需求也在提高。具体来说,第五
代通信系统提出的低时延、高带宽、海量接入的要求,是现代卫星通信系统必须面临的
考验。相较于地面站海量的通信需求,如何充分利用卫星转发器有限的功率和频带,是
卫星通信的一个重要问题。尽管通信信道的资源分配问题已有一些经典算法被提出,但
是低轨道卫星系统网络结构复杂,移动性强,需要具备星间通信的能力,存在很多传统
地面通信中未曾遇到的困难和挑战。因此,新的资源分配方法亟待研究,而既存的传统
算法也需要推广和改进,来适应日新月异的通信需求。\par%体制

虽然低轨通信卫星通信面临着前所未有的挑战,但是通信卫星的质量也在提升,这为更
高效、更灵活的资源分配算法提供了硬件基础。过去卫星通信常用C、Ku频段,近年来
已经扩展为Ka频段,甚至到V频段,相应的,可用带宽从500MHz,直到现在的2GHz,再
加上多波束频率复用和极化复用等技术,单颗卫星的可用带宽已经可以达到数十GHz,
将卫星通信的业务拓展到多媒体领域。值得一提的是,考虑到电离层的影响,卫星通信
的频率不能过低;考虑到云、雾、雨的吸收作用,卫星通信的信号频率也不能过高。所
以,尽管带宽有所提高,但决不是无限的,仍然需要高效的资源分配算法来合理分配。
另一方面,天线技术也取得了长足的进步。最初的静止轨道卫星天线采用覆球单波束覆
盖,这种波束虽然简单,但是不够灵活,信道容量低;波束赋形天线则可以产生与被覆
盖区域的地理外形相吻合的主波束形状;更具吸引力的技术是多波束天线,采用多重频
率复用蜂窝状多点波束覆盖,可以进行实现频率复用,使通信容量成倍增加,还可以根
据需要实现波束扫描或波束重构,以适应对地覆盖区域的变化,从而使系统具有很大的
灵活性。\par%频率 ,天线

多波束卫星通信系统使用了频率复用的思想,可以成倍地增加信道带宽,但是相邻波束
之间的同频干扰会极大地影响通信质量。预编码技术应运而生。地面网关可以获知信道
状态信息,并以此为依据计算出预编码矩阵,将待发送信号预编码后再传输给卫星。预
编码矩阵可以抵消信道的干扰,从而显著提高接收信干噪比,使得服务质量大幅提高。
然而,预编码矩阵的计算复杂度是很高的,与波束个数的平方成正比。因此,亟需研究
低复杂度的预编码算法,才能将预编码的思想应用于实际工程中。\par%预编码

多波束卫星不但由于采用频率复用技术存在着的不可避免的波束间干扰问题,而且无
法根据地面波位流量需求的变化实时调整星上资源的分配,这种资源分配的不灵活
性将会导致卫星资源调度效率的下降,引起通信资源的浪费,势必会成为限制卫星通信
性能的重要瓶颈。针对以上问题,研究人员提出了基于时间分片的跳波束通信方案,
与传统的多波束卫星通信系统相比,基于跳波束技术的多波束卫星具有调度灵活、系统
容量大、频谱效率高等优点,其基本思想是利用时间分片技术,在同一时间只有一部分
波束进行服务,可以实现对空间、时间、带宽、功率四个维度的资源分配,能够灵活适
应地面业务需求量的动态变化,可大大降低地面通信关口站的成本,因此研究跳波束技
术不仅具有科研意义,而且具有实际应用价值。跳波束技术之前受制于天线技术的发
展,现在逐渐被研究人员重视并发展起来,所以对于基于跳波束技术的卫星通信系
统,在资源分配和跳波束图案设计等方面均可进行深入研究,以有效抑制系统内同频干
扰,大幅度提高系统吞吐量和服务满意度,使得实际分配给各个波束的系统容量能够更
好地满足动态变化和不均匀分布的地面波位业务需求。\par%跳波束
\section{国内外研究现状}
\label{sec:template_download}
本节介绍卫星通信系统、低轨多波束卫星预编码算法和跳波束算法的国内外研究现状。\par
\subsection{卫星通信研究现状}
静止轨道卫星通信系统已经成熟应用,而低轨卫星通信系统正在蓬勃发展。国外LEO 
卫星通信系统的建设要早于国内,主要有Teledesic、铱星和全球星系统等。文献描述
了 Teledesic 卫星系统;铱星系统是由摩托罗拉公司设计的美国第一代卫星移动通信
星座系统,文献对其进行了全面描述;全球星系统是美国高通公司设计的卫星移动通信
系统,文献从系统架构、终端设计和成本分析等方面对比了以上三种卫星系统,但面对
日益增长的业务需求,这些卫星通信系统难以满足人们的通信需求。进入二十一世纪之
后,全球商业和军事领域对 LEO 卫星移动通信系统的研究日益深入,产生了一批构建 
LEO 卫星星座的新计划和新技术,其中最为代表性的是美国 SpaceX、OneWeb 和 
Telesat 低轨卫星通信系统。\par
\subsection{预编码技术研究现状}

\subsection{跳波束研究现状}
随着相控阵天线的发展,研究人员在多波束卫星技术的基础上发展出跳波束技术,
它能够充分利用波束的时间和空间自由度 [42],并且在第二代卫星数字视频广播扩展版
(Digital Video Broadcasting-Satellite-Second Generation Extensions,DVB-S2X)标准中提
出了几种超帧结构以支持未来多波束卫星中应用跳波束技术 [43],这意味着跳波束技术
具有巨大潜力。跳波束卫星和传统多波束卫星的性能已被广泛研究,文献 [44] 分析了基
于跳波束技术的宽带 LEO 卫星网络的下行吞吐量,并给出了下行吞吐量的上界和下界。
文献 [9] 重点研究了跳波束卫星为流量需求分布稀疏地区的用户提供多媒体接入的应用
场景,与传统多波束卫星系统相比,基于跳波束技术的卫星通信系统实现了约 30% 的
吞吐量增长。文献 [45] 侧重于跳波束卫星前向链路的传输性能,仿真结果表明,跳波束
卫星系统在匹配流量需求以及资源利用率方面都优于传统的多波束卫星系统。\par

基于跳波束技术的多波束卫星通信系统中的资源分配算法也是一个研究热点。文
献 [46] 梳理了现有的多波束卫星的资源分配算法,验证了跳波束技术相比于传统多波
束技术在卫星资源分配方面的优越性。但现有的跳波束技术研究主要集中在 GEO 卫星
上,由于 LEO 卫星上更有限的带宽和功率资源以及较快的移动速度,GEO 卫星上适用的
跳波束技术不能直接应用于 LEO 卫星系统,因此研究人员探讨了跳波束技术在 LEO
卫星系统下的自适应改进方案。文献 [47] 首次将跳波束技术引入到 LEO 卫星系统资源
分配算法中,并通过吞吐量、时延等系统指标分析了所提出算法的有效性。文献 [48] 将
贪心算法应用于跳波束卫星资源分配中,从吞吐量和服务成功率对算法性能进行了评
估。文献 [44, 49] 构建了以最大化吞吐量和最小化传输时延为目标的资源分配模型,但
只考虑了单一维度的资源分配优化,没有提出完整的多维度优化模型,因此文献 [50] 研
究了以吞吐量最大化和传输时延最小化为目标的多维资源联合分配问题,并使用启发式
算法对问题进行求解。文献 [51] 关注了跳波束卫星系统中的时延公平性问题,提出了一
种面向时延公平的资源分配算法。文献 [52] 则对现有的启发式算法进行改进,提出了一
种高效的多维资源分配机制,通过改进的遗传算法灵活分配带宽和功率资源,在系统吞
吐量、服务成功率和传输时延方面取得了较好的性能。\par

随着人工智能技术的发展,深度学习被视为处理复杂资源分配的一种有效方
法 [53–56],因其能够以低复杂度保证高性能已被应用于跳波束卫星系统的资源分配算法
中。文献 [53] 提出了一种结合深度学习的资源分配优化算法,可以快速生成可行的跳
波束图案,但此方法采用了监督学习,对数据样本有很强的依赖性。因为跳波束卫星的
资源分配过程可以看做一个顺序决策问题,文献 [54] 将基于跳波束技术的资源分配过
程构建为一个多目标马尔科夫决策过程,并使用深度强化学习来处理,但只设计了跳波
束图案并且假设所有波束共享全部带宽,将导致波束间严重的同频干扰问题。文献 [57]
对现有算法提出改进,提出了一种多智能体深度强化学习算法,以最大化吞吐量和时延
公平性为目标对跳波束图案进行设计,并根据流量需求灵活分配带宽和功率资源,获得
了良好的系统性能,因此在跳波束卫星通信系统中,结合人工智能技术的资源分配算法
具有广阔的研究前景。\par







